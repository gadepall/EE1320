%The Raspberry Pi has 40 GPIO pins \cite{gpio_pins}, as shown in the figure below.  The Wiring Pi \cite{wiringpi}  GPIO access library allows for an Arduino style access to the GPIO pins for input/output. The pins 0-6 and 10-14 can be used for digital I/O.
%\end{center}
%%Out of these, some are  ground pins, some are 5$V$ and 3.3$V$ pins and others input/output pins.  Out of these input/output  analog input pins A0-A3 and digital pins D1-D13 that can be used for both input as well as output. It also has two power pins that can generate 3.3$V$ and 5$V$.  In the following exercises, only the GND, 5$V$ and digital pins will be used.
%%
%%

%\subsection{GPIO pin connection}
%Open the blink.c program arduino software.  Check if the ports show Arduino Uno and click the appropriate button.  Open Examples$\rightarrow$Basics$\rightarrow$Blink.  


%
\subsection{Counting Decoder}
%
Table \ref{table:counter_decoder} represents the system that increments the numbers 0-8 by 1 and resets the number 9 to 0. Such a table is known as a truth table, where $W,X,Y,Z$ are the inputs
and $A,B,C,D$ are the outputs. 
Note that  $D = 1$ for the inputs $0111$ and $1000$.  Using {\em boolean} logic,
%
\begin{equation}
\label{bool_logic}
D = WXYZ^{'} + W^{'}X^{'}Y^{'}Z
\end{equation}
%
Note that $0111$ results in the expression $WXYZ^{'}$ and $1000$ yields $W^{'}X^{'}Y^{'}Z$. 
\begin{problem}
	\label{counter_dec}
Write the boolean logic functions for $A,B,C$ in terms of $W,X,Y,Z$.
\end{problem}
%%
		\input{./chapter3/tables/counter_decoder}
%\onecolumn
%%\setcounter{problem}{6}  
%\input{./chapter3/tables/counter_decoder}
%%
%%\setcounter{problem}{0}  
%\twocolumn

%\begin{table}
%\end{table}
The $\&\&$ operand is used for the boolean AND (multiplication) operation, the $||$ operand is used for the OR (addition) operation and the ! operand is used for the NOT ($^{'}$) operation in Arduino code.  For example, the expression for \eqref{bool_logic} in Arudino is
\begin{verbatim}
D = (W&&X&&Y&&!Z)||(!W&&!X&&!Y&&Z);
\end{verbatim}
%
\begin{problem}
Write the code for the outputs $A,B,C$ and verify if your logic is correct by observing the output on the seven segment display.
\end{problem}
%
\subsection{Display Decoder}
%
\begin{problem}
Now write the truth table for the seven segment display decoder (IC 7447).  The inputs will be $A,B,C,D$ and the outputs will be $a,b,c,d,e,f,g$.
\end{problem}
%
\begin{problem}
\label{seven_seg_disp_logic}
Obtain the logic functions for outputs $a,b,c,d,e,f,g$ in terms of the inputs $A,B,C,D$.
\end{problem}
\begin{problem}
Disconnect the Pi from IC 7447 and connect the pins 0-6 in Fig. \ref{fig_1_3b} in the Pi directly to the seven segment display.
\end{problem}
\begin{problem}
Write a new program to implement the logic in Problem \ref{seven_seg_disp_logic} and observe the output in the display.  You have designed the logic for IC 7447!
\end{problem}
\begin{problem}
Now include your counting decoder program in the  display decoder program
and see if the display shows the consecutive number.
\end{problem}
\subsection{Software Counter}
%
\begin{problem}
Connect pin 10 to the dot pin of the display and execute the following program.  What do you observe?
\end{problem}
%
\lstinputlisting{./codes/blink.c}
%
\begin{problem}
A decade counter counts the numbers from 0-9 and then resets to 0.  Suitable modify the above program to obtain a decade counter.
\end{problem}

%\begin{figure*}
%\setcounter{problem}{1}  
%\onecolumn
		%\input{./chapter3/tables/counter_decoder}
%\twocolumn
%\setcounter{problem}{8}  
%\end{figure*}

%%\begin{problem}
%%Generate the boolean functions for the segments $a-f$ using the table in Problem \ref{bcd_ss}.  For example, the function for $a$ is obtained from the table as
%%\begin{equation}
%%a=\bar{D}\bar{C}\bar{B}A+\bar{D}C\bar{B}\bar{A}
%%\label{boolean}
%%\end{equation}
%%\end{problem}
%%%
%%\begin{problem}
	%%\label{counter_dec}
%%Write functions for $A,B,C,D$ in Arduino using the following table and verify using the Arduino driven display.
		%%\input{counter_decoder}
%%\end{problem}
%%\begin{problem}
	%%Write a module for decimal to binary conversion
	%%according to the example given below
	%%\input{conversion}
	%%%
	%%$N \% 2$ gives the remainder and $N/2$ gives the quotient
%%	and use it in the above code so that decimal values are given as input in the program and observed as output in the display. Note that the following code
%%	\begin{verbatim}
%%	a % b
%%	\end{verbatim}
%%	can be used to obtain the remainder when a is divided by b and
%%	\begin{verbatim}
%%	a/b
%%	\end{verbatim}
%%	gives the quotient.
%%\end{problem}
