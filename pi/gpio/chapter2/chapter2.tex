
\begin{problem}
Connect the A-D pins of the 7447 IC  in Fig. \ref{fig_1_9} to the GPIO pins  0-3 of the Pi shown in Figs. \ref{fig_1_3a} and \ref{fig_1_3b}.
\end{problem}	
%
\begin{problem}
Execute the following code using the following commands:
\\
gcc -Wall -o test bcd\_seven.c -lwiringPi
\\
sudo ./test
\\
What do you observe?
\lstinputlisting{./codes/bcd_seven.c}
\end{problem}
\begin{problem}
Now generate the numbers 0-9 by modifying the above program.
\end{problem}
\begin{problem}
After the following line in the previous code,
\begin{verbatim}
int main (void) {
\end{verbatim}
you can define integer variables as
\begin{verbatim}
int A = 0;
\end{verbatim}	
where the variable A is defined to be an integer and given the value 0.  Define variables A,B,C,D as 0 or 1 and use the digitalWrite() command as in the earlier code to generate the numbers 0-9.
\end{problem}
%%
