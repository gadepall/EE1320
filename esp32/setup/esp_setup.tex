\documentclass[journal,12pt,twocolumn]{IEEEtran}
%
\usepackage{setspace}
\usepackage{gensymb}
\usepackage{xcolor}
\usepackage{caption}
%\usepackage{subcaption}
%\doublespacing
\singlespacing

%\usepackage{graphicx}
%\usepackage{amssymb}
%\usepackage{relsize}
\usepackage[cmex10]{amsmath}
\usepackage{mathtools}
%\usepackage{amsthm}
%\interdisplaylinepenalty=2500
%\savesymbol{iint}
%\usepackage{txfonts}
%\restoresymbol{TXF}{iint}
%\usepackage{wasysym}
\usepackage{amsthm}
\usepackage{mathrsfs}
\usepackage{txfonts}
\usepackage{stfloats}
\usepackage{cite}
\usepackage{cases}
\usepackage{subfig}
%\usepackage{xtab}
\usepackage{longtable}
\usepackage{multirow}
%\usepackage{algorithm}
%\usepackage{algpseudocode}
\usepackage{enumitem}
\usepackage{mathtools}
\usepackage{hyperref}

%\usepackage[framemethod=tikz]{mdframed}
\usepackage{listings}
    \usepackage[latin1]{inputenc}                                 %%
    \usepackage{color}                                            %%
    \usepackage{array}                                            %%
    \usepackage{longtable}                                        %%
    \usepackage{calc}                                             %%
    \usepackage{multirow}                                         %%
    \usepackage{hhline}                                           %%
    \usepackage{ifthen}                                           %%
  %optionally (for landscape tables embedded in another document): %%
    \usepackage{lscape}     

\usepackage{iithtlc}
\usepackage{tikz}
\usepackage{circuitikz}
\usepackage{url}
\def\UrlBreaks{\do\/\do-}


%\usepackage{stmaryrd}


%\usepackage{wasysym}
%\newcounter{MYtempeqncnt}
\DeclareMathOperator*{\Res}{Res}
%\renewcommand{\baselinestretch}{2}
\renewcommand\thesection{\arabic{section}}
\renewcommand\thesubsection{\thesection.\arabic{subsection}}
\renewcommand\thesubsubsection{\thesubsection.\arabic{subsubsection}}

\renewcommand\thesectiondis{\arabic{section}}
\renewcommand\thesubsectiondis{\thesectiondis.\arabic{subsection}}
\renewcommand\thesubsubsectiondis{\thesubsectiondis.\arabic{subsubsection}}

% correct bad hyphenation here
\hyphenation{op-tical net-works semi-conduc-tor}

\lstset{
language=C,
frame=single, 
breaklines=true
}

%\lstset{
	%%basicstyle=\small\ttfamily\bfseries,
	%%numberstyle=\small\ttfamily,
	%language=Octave,
	%backgroundcolor=\color{white},
	%%frame=single,
	%%keywordstyle=\bfseries,
	%%breaklines=true,
	%%showstringspaces=false,
	%%xleftmargin=-10mm,
	%%aboveskip=-1mm,
	%%belowskip=0mm
%}

%\surroundwithmdframed[width=\columnwidth]{lstlisting}
\def\inputGnumericTable{}                                 %%
\lstset{
language=C,
frame=single, 
breaklines=true
}
 

\begin{document}
%

\theoremstyle{definition}
\newtheorem{theorem}{Theorem}[section]
\newtheorem{problem}{Problem}
\newtheorem{proposition}{Proposition}[section]
\newtheorem{lemma}{Lemma}[section]
\newtheorem{corollary}[theorem]{Corollary}
\newtheorem{example}{Example}[section]
\newtheorem{definition}{Definition}[section]
%\newtheorem{algorithm}{Algorithm}[section]
%\newtheorem{cor}{Corollary}
\newcommand{\BEQA}{\begin{eqnarray}}
\newcommand{\EEQA}{\end{eqnarray}}
\newcommand{\define}{\stackrel{\triangle}{=}}

\bibliographystyle{IEEEtran}
%\bibliographystyle{ieeetr}

\providecommand{\nCr}[2]{\,^{#1}C_{#2}} % nCr
\providecommand{\nPr}[2]{\,^{#1}P_{#2}} % nPr
\providecommand{\mbf}{\mathbf}
\providecommand{\pr}[1]{\ensuremath{\Pr\left(#1\right)}}
\providecommand{\qfunc}[1]{\ensuremath{Q\left(#1\right)}}
\providecommand{\sbrak}[1]{\ensuremath{{}\left[#1\right]}}
\providecommand{\lsbrak}[1]{\ensuremath{{}\left[#1\right.}}
\providecommand{\rsbrak}[1]{\ensuremath{{}\left.#1\right]}}
\providecommand{\brak}[1]{\ensuremath{\left(#1\right)}}
\providecommand{\lbrak}[1]{\ensuremath{\left(#1\right.}}
\providecommand{\rbrak}[1]{\ensuremath{\left.#1\right)}}
\providecommand{\cbrak}[1]{\ensuremath{\left\{#1\right\}}}
\providecommand{\lcbrak}[1]{\ensuremath{\left\{#1\right.}}
\providecommand{\rcbrak}[1]{\ensuremath{\left.#1\right\}}}
\theoremstyle{remark}
\newtheorem{rem}{Remark}
\newcommand{\sgn}{\mathop{\mathrm{sgn}}}
\providecommand{\abs}[1]{\left\vert#1\right\vert}
\providecommand{\res}[1]{\Res\displaylimits_{#1}} 
\providecommand{\norm}[1]{\lVert#1\rVert}
\providecommand{\mtx}[1]{\mathbf{#1}}
\providecommand{\mean}[1]{E\left[ #1 \right]}
\providecommand{\fourier}{\overset{\mathcal{F}}{ \rightleftharpoons}}
%\providecommand{\hilbert}{\overset{\mathcal{H}}{ \rightleftharpoons}}
\providecommand{\system}{\overset{\mathcal{H}}{ \longleftrightarrow}}
	%\newcommand{\solution}[2]{\textbf{Solution:}{#1}}
\newcommand{\solution}{\noindent \textbf{Solution: }}
\providecommand{\dec}[2]{\ensuremath{\overset{#1}{\underset{#2}{\gtrless}}}}
%\numberwithin{equation}{subsection}
\numberwithin{equation}{problem}
%\numberwithin{problem}{subsection}
%\numberwithin{definition}{subsection}
\makeatletter
\@addtoreset{figure}{problem}
\makeatother

\let\StandardTheFigure\thefigure
%\renewcommand{\thefigure}{\theproblem.\arabic{figure}}
\renewcommand{\thefigure}{\theproblem}


%\numberwithin{figure}{subsection}

%\numberwithin{equation}{subsection}
%\numberwithin{equation}{section}
%%\numberwithin{equation}{problem}
%%\numberwithin{problem}{subsection}
%\numberwithin{problem}{section}
%%\numberwithin{definition}{subsection}
%\makeatletter
%\@addtoreset{figure}{problem}
%\makeatother
%\makeatletter
%\@addtoreset{table}{problem}
%\makeatother

%\let\StandardTheFigure\thefigure
%\let\StandardTheTable\thetable
%%\renewcommand{\thefigure}{\theproblem.\arabic{figure}}
%\renewcommand{\thefigure}{\theproblem}
%\renewcommand{\thetable}{\theproblem}
%%\numberwithin{figure}{section}

%%\numberwithin{figure}{subsection}



\def\putbox#1#2#3{\makebox[0in][l]{\makebox[#1][l]{}\raisebox{\baselineskip}[0in][0in]{\raisebox{#2}[0in][0in]{#3}}}}
     \def\rightbox#1{\makebox[0in][r]{#1}}
     \def\centbox#1{\makebox[0in]{#1}}
     \def\topbox#1{\raisebox{-\baselineskip}[0in][0in]{#1}}
     \def\midbox#1{\raisebox{-0.5\baselineskip}[0in][0in]{#1}}

\vspace{3cm}

\title{ 
	\logo{
	Software Setup for ESP32
	}
}
%\title{
%	\logo{Matrix Analysis through Octave}{\begin{center}\includegraphics[scale=.24]{tlc}\end{center}}{}{HAMDSP}
%}


% paper title
% can use linebreaks \\ within to get better formatting as desired
%\title{Matrix Analysis through Octave}
%
%
% author names and IEEE memberships
% note positions of commas and nonbreaking spaces ( ~ ) LaTeX will not break
% a structure at a ~ so this keeps an author's name from being broken across
% two lines.
% use \thanks{} to gain access to the first footnote area
% a separate \thanks must be used for each paragraph as LaTeX2e's \thanks
% was not built to handle multiple paragraphs
%

\author{N Chakradhar$^{*}$, P.~N.~V.~S.~S.~K.~ HAVISH$^{*}$, S.~.S.~Ashish$^{*}$ and G V V Sharma$^{*}$ %<-this  stops a space
\thanks{*The authors are with the Department
of Electrical Engineering, Indian Institute of Technology, Hyderabad
502285 India e-mail:  gadepall@iith.ac.in.  All material in this manual is released under GNU GPL. Free to use for anything.}% <-this % stops a space
%\thanks{J. Doe and J. Doe are with Anonymous University.}% <-this % stops a space
%\thanks{Manuscript received April 19, 2005; revised January 11, 2007.}}
}
% note the % following the last \IEEEmembership and also \thanks - 
% these prevent an unwanted space from occurring between the last author name
% and the end of the author line. i.e., if you had this:
% 
% \author{....lastname \thanks{...} \thanks{...} }
%                     ^------------^------------^----Do not want these spaces!
%
% a space would be appended to the last name and could cause every name on that
% line to be shifted left slightly. This is one of those "LaTeX things". For
% instance, "\textbf{A} \textbf{B}" will typeset as "A B" not "AB". To get
% "AB" then you have to do: "\textbf{A}\textbf{B}"
% \thanks is no different in this regard, so shield the last } of each \thanks
% that ends a line with a % and do not let a space in before the next \thanks.
% Spaces after \IEEEmembership other than the last one are OK (and needed) as
% you are supposed to have spaces between the names. For what it is worth,
% this is a minor point as most people would not even notice if the said evil
% space somehow managed to creep in.



% The paper headers
%\markboth{Journal of \LaTeX\ Class Files,~Vol.~6, No.~1, January~2007}%
%{Shell \MakeLowercase{\textit{et al.}}: Bare Demo of IEEEtran.cls for Journals}
% The only time the second header will appear is for the odd numbered pages
% after the title page when using the twoside option.
% 
% *** Note that you probably will NOT want to include the author's ***
% *** name in the headers of peer review papers.                   ***
% You can use \ifCLASSOPTIONpeerreview for conditional compilation here if
% you desire.




% If you want to put a publisher's ID mark on the page you can do it like
% this:
%\IEEEpubid{0000--0000/00\$00.00~\copyright~2007 IEEE}
% Remember, if you use this you must call \IEEEpubidadjcol in the second
% column for its text to clear the IEEEpubid mark.



% make the title area
%\maketitle

%\newpage

%\tableofcontents


%\begin{abstract}
%%\boldmath
%In this letter, an algorithm for evaluating the exact analytical bit error rate  (BER)  for the piecewise linear (PL) combiner for  multiple relays is presented. Previous results were available only for upto three relays. The algorithm is unique in the sense that  the actual mathematical expressions, that are prohibitively large, need not be explicitly obtained. The diversity gain due to multiple relays is shown through plots of the analytical BER, well supported by simulations. 
%
%\end{abstract}
% IEEEtran.cls defaults to using nonbold math in the Abstract.
% This preserves the distinction between vectors and scalars. However,
% if the journal you are submitting to favors bold math in the abstract,
% then you can use LaTeX's standard command \boldmath at the very start
% of the abstract to achieve this. Many IEEE journals frown on math
% in the abstract anyway.

% Note that keywords are not normally used for peerreview papers.
%\begin{IEEEkeywords}
%Cooperative diversity, decode and forward, piecewise linear
%\end{IEEEkeywords}



% For peer review papers, you can put extra information on the cover
% page as needed:
% \ifCLASSOPTIONpeerreview
% \begin{center} \bfseries EDICS Category: 3-BBND \end{center}
% \fi
%
% For peerreview papers, this IEEEtran command inserts a page break and
% creates the second title. It will be ignored for other modes.
\IEEEpeerreviewmaketitle


%\documentclass{article}
%\usepackage[utf8]{inputenc}
%\usepackage{listings}
%\usepackage{graphicx}

%\usepackage{color}
%\definecolor{codegreen}{rgb}{0,0.6,0}
%\definecolor{codegray}{rgb}{0.5,0.5,0.5}
%\definecolor{codepurple}{rgb}{0.58,0,0.82}
%\definecolor{backcolour}{rgb}{0.95,0.95,0.92}
%\lstdefinestyle{mystyle}{
    %backgroundcolor=\color{backcolour},   
    %commentstyle=\color{codegreen},
    %keywordstyle=\color{magenta},
    %numberstyle=\tiny\color{codegray},
    %stringstyle=\color{codepurple},
    %basicstyle=\footnotesize,
    %breakatwhitespace=false,         
    %breaklines=true,                 
    %captionpos=b,                    
    %keepspaces=true,                 
    %numbers=left,                    
    %numbersep=5pt,                  
    %showspaces=false,                
    %showstringspaces=false,
    %showtabs=false,                  
    %tabsize=2
%}
 
%\lstset{style=mystyle}


%\title{Analog Design Through ESP32}
%\author{G V V Sharma* }

%\begin{document}

\maketitle
\begin{abstract}
This manual shows how to setup the software tools for ESP32 on Raspbian OS.  The process is similar for any Debian style  OS.
\end{abstract}
%\section{Components}
%\input{./figs/components.tex}
%\section{Pre-requisites}
%Before starting anything, make sure the Debian system is up-to-date and you have Python2.7 and Python3 installed on your Linux
%\begin{lstlisting}
%sudo apt-get update
%sudo apt-get upgrade
%\end{lstlisting}
\section{Firmware}

%Firmware is an essential software programmed into a device to provide a low-level control to the device's specific hardware. In ESP32's case, we download the firmware from official website to prevent any future issues.\newline
\begin{enumerate}



\item Go to \href{http://micropython.org/download}{ \url{http://micropython.org/download}}, scroll down to ESP32 downloads list and download the latest firmware for ESP32 boards (.bin file).  

\item Now install \textbf{esptool}
\begin{lstlisting}
sudo pip install esptool 
\end{lstlisting}
\end{enumerate}
%
%
%\begin{figure}[!h]
%\centering
%\includegraphics[width=\columnwidth]{./figs/breadboard.eps}
%\caption{Breadboard}
%\label{fig:breadboard}
%\end{figure}
%
%\section{ESPTOOL}
%\begin{problem}
%We use a tool called esptool - a python script used to flash the firmware to ESP32. Type the following commands in Linux terminal to setup esptool

%\end{problem}
%
%
\section{Connecting ESP32}
%\begin{problem}
\begin{enumerate}


\item Connect the ESP32 to system and type the following commands to know which port it is connected to.
\begin{lstlisting}
dmesg | grep tty
\end{lstlisting}
Suppose it is connected to ttyUSB0. Go to the folder where the .bin was saved and execute 
\begin{lstlisting}
esptool.py --port /dev/ttyUSB0 erase_flash
esptool.py --chip esp32 --port /dev/ttyUSB0 write_flash -z 0x1000 filename.bin
\end{lstlisting}
\end{enumerate}
%If the flashing process was successful, the output should be similar to this form
%\begin{lstlisting}
%esptool.py v2.1
%Connecting........_
%Chip is unknown ESP32 (revision 0)
%Uploading stub...
%Running stub...
%Stub running...
%Configuring flash size...
%Auto-detected Flash size: 4MB
%Compressed 903392 bytes to 567400...
%Wrote 903392 bytes (567400 compressed) at 0x00001000 in 50.1 seconds (effective 144.3 kbit/s)...
%Hash of data verified.
%
%Leaving...
%Hard resetting...
%\end{lstlisting}
%\end{problem}
%
%
%\begin{figure}
%\centering
%\includegraphics[width=\columnwidth]{./figs/voltage_divider.eps}
%\resizebox {\columnwidth} {!} {
%\input{./figs/collpits.tex}
%\input{./figs/vrr.tex}
%}
%\input{./figs/vrr.tex}
%\includegraphics[width=\columnwidth]{./figs/voltage_divider.eps}
%\caption{Voltage Divider}
%\label{fig:voltage_divider}
%\end{figure}
%
\section{SCREEN}
%\begin{problem}
To test the firmware, we can use the screen command to communicate to the board directly. To do it, use these series of commands.
\begin{enumerate}
\item

\begin{lstlisting}
sudo apt-get install screen
screen /dev/ttyUSB0 115200
\end{lstlisting}
Press enter key. You will be presented with a python-like terminal where in you can give python commands directly.
\item
To test if it's working or not, just use this command

\begin{lstlisting}
print ("Hello_World")
\end{lstlisting}
The terminal output should be similar to this:
\begin{lstlisting}
>>> print ("Hello_World")
Hello_world
>>>
\end{lstlisting}
\end{enumerate}
%\end{problem}
%
%\begin{problem}
%\newpage
\section{AMPY}
\textbf{ampy} is used to upload or download codes from the ESP32.
%\newline
To get this tool, execute the following command
\begin{lstlisting}
sudo pip install adafruit-ampy 
\end{lstlisting} 
\section{Arduino IDE}
\begin{enumerate}
\item Download the latest Arduino IDE from \href{https://www.arduino.cc/en/Main/Software}{https://www.arduino.cc/en/Main/Software}
\item For making the ESP32 board visible in the Arduino IDE, the following instructions have to be executed.  Make sure that you are doing this using a downloaded IDE.
\begin{lstlisting}
cd ~/Arduino/hardware/espressif && \
git clone https://github.com/espressif/arduino-esp32.git esp32 && \
cd esp32 && \
git submodule update --init --recursive && \
cd tools && \
python get.py
\end{lstlisting}

\end{enumerate}
\section{XTENSA for RPI 3}
Download the {\em xtensa} library for esp32 from
\href{http://tlc.iith.ac.in/resources/xtensa-rpi.7z}{http://tlc.iith.ac.in/resources/xtensa-rpi.7z}
or
\href{http://tlc.iith.ac.in/resources/xtensa-toolchain-rpi.zip}{http://tlc.iith.ac.in/resources/xtensa-toolchain-rpi.zip}
%\begin{lstlisting}
%https://drive.google.com/drive/folders/0Bz7Qo_vFhiBmVFYyQlA3ZWxDZ0U?usp=sharing
%\end{lstlisting}
and extract it to 
\begin{lstlisting}
/home/pi/Arduino/hardware/espressif/esp32/tools 
\end{lstlisting}
directory.
You can also build the xtensa library through the following commands
\begin{lstlisting}
sudo apt-get install gawk gperf grep gettext automake bison flex texinfo help2man libtool libtool-bin git wget make libncurses-dev python python-serial python-dev python-pip
sudo pip install pyserial
cd ~/esp
git clone -b xtensa-1.22.x https://github.com/espressif/crosstool-NG.git
cd crosstool-NG
./bootstrap && ./configure --enable-local && make install
./ct-ng xtensa-esp32-elf
nano ./.config
-- > Find CT_PARALLEL_JOBS=0 and change 0 to 1
./ct-ng build 
chmod -R u+w builds/xtensa-esp32-elf
\end{lstlisting}
\section{Blink Program}
\begin{enumerate}
\item Connect the ESP 32 to the computer.
\item Run the following code.  You should see the onboard LED blinking.
\lstinputlisting{./codes/hello/hello.ino}
\end{enumerate}
%Make the following connections
%

%If this step gives you any errors, try these instead.
%\begin{lstlisting}
%sudo pip install adafruit-ampy --upgrade
%ampy --version
%ampy, version 1.0.2
%\end{lstlisting}
%\end{problem}
%
%\section{Uploading codes}
%Route your terminal to the directory where you saved your python code and execute the following command run it on ESP32 (without uploading)
%
%\begin{lstlisting}
%ampy --port /dev/ttyUSB0 run codename.py
%\end{lstlisting}
%%\newline
%Similarly to upload the code
%\begin{lstlisting}
%ampy --port /dev/ttyUSB0 put codename.py
%\end{lstlisting}
%Similarly to copy the code from ESP32
%\begin{lstlisting}
%ampy --port /dev/ttyUSB0 get codename.py
%\end{lstlisting}
%Replace $codename.py$ with your code name. Tab prompting will work here.
%
%\lstinputlisting{./codes/resistance/src/main.cpp}
%
%\begin{problem}
%Now keep $R_1 = 1 K$ and $R_2=220 \Omega$ and verify by modifying the above code.
%\end{problem}
%\section{Display Resistance on LCD}
%%
%\begin{problem}
%Plug the LCD in Fig. \ref{fig:lcd} to the breadboard.
%\end{problem}
%%
%\begin{figure}
%\centering
%\includegraphics[width=\columnwidth]{./figs/lcd.eps}
%\caption{lcd}
%\label{fig:lcd}
%\end{figure}
%%
%\begin{problem}
%Connect the 220$\Omega$ resistance from $V_{cc}$ to pin 15 (Led+) of the LCD.
%\end{problem}
%%
%%%
%%\begin{problem}
%%Connect  pin 16 (Led-) of the LCD to GND.  The LCD should glow.
%%\end{problem}
%%%
%%\begin{problem}
%%Connect pin 3 of the LCD to GND.  This is required for contrast.
%%\end{problem}
%\begin{problem}
%Connect the Arduino pins to LCD pins as per Table \ref{table:lcd pins}.
%%\begin{center}
%%\begin{tabular}{ |p{3cm}|p{3cm}|}
% %\hline
% %\multicolumn{2}{|c|}{\textbf{ESP32 to LCD Connection}} \\
% %\hline
% %\textbf{ESP32 Pins} & \textbf{LCD Pins} \\
% %\hline
% %Pin D2 & Pin D7\\
% %\hline
% %Pin D3 & Pin D6\\
% %\hline
% %Pin D4 & Pin D5\\
% %\hline
% %Pin D5 & Pin D4\\
% %\hline
% %Pin 11 & Pin 6\\
% %\hline
% %Pin 12 & Pin 4\\
% %\hline
% %GND Pin & Pin 1 and Pin 5\\
% %\hline
% %5V  Pin & Pin 2\\
% %\hline
%%\end{tabular}
%%\end{center}
%\end{problem}
%%
%\input{./figs/table.tex}
%%%
%%\begin{problem}
%%The Potentiometer to LCD connection are given below: 
%%\begin{center}
%%\begin{tabular}{ |p{3cm}|p{3cm}|}
% %\hline
% %\multicolumn{2}{|c|}{\textbf{Potentiometer to LCD Connection}} \\
% %\hline
% %\textbf{Potentiometer} & \textbf{LCD Pins} \\
% %\hline
% %Middle Pin & Pin 3\\
% %\hline
% %Pin 1 & Pin 2(Vcc Pin)\\
% %\hline
% %Pin 3 & Pin 1(GND Pin)\\
% %\hline
%%\end{tabular}
%%\end{center}
%%\end{problem}
%%
%%
%%\begin{problem}
%%The resistance is an analogue function,but the value displayed on LCD is digital function.So,we need to do analogue to digital conversion,ESP32 has built-in 10-bit analogue to digital converter.
%%Two resistors are used, one of which is known resistor value and other is unknown resistor value.
%%Take 2 resistors,one known resistor with resistance value(For e.g-1K,2K,10K),here we have taken 1K and another an unknown resistor whose value we are going to calculate.
%%\end{problem}
%%
%%
%\begin{problem}
%Include the instructions for the LCD in the code for measuring the resistance. 
%\end{problem}
%\solution 
%%
%\lstinputlisting{./codes/lcd/resistance_lcd.ino}
%\section{Miscellaneous}
%Various other functions and their usage has been documented and is present in the official \href{http://docs.micropython.org/en/v1.9.3/esp8266/index.html}{http://docs.micropython.org/en/v1.9.3/esp8266/index.html}. 



%\begin{enumerate}


%\item We create a variable called analogPin and assign it to 0. 
%This is because the voltage value we are going to read is connected to analogPin %GPIO32.



%\item  The 12-bit ADC can differentiate 4096 discrete voltage levels, 5 volt is applied to 2 resistors and the voltage sample is taken in between the resistors. The value which we get from analogPin can be between 0 and 4095. 0 would represent 0 volts falls across the unknown resistor. A value of 4095 would mean that practically all 5 volts falls across the unknown resistor.



%\item  $V_{out}$ represents the divided voltage that falls across the unknown resistor.



%\item  The Ohm meter in this manual works on the principle of the voltage divider shown in Fig. \ref{fig:voltage_divider}.
%
%\begin{align}
%V_{out}&=\frac{R_1}{R_1+R_2}V_{in} \\
%\Rightarrow R_2&=R_1\brak{\frac{V_{in}}{V_{out}}-1}
%\end{align}
%
%In the above, $V_{in} = 5$V, $R_1 = 220 \Omega$.
%\end{enumerate}


\end{document}


